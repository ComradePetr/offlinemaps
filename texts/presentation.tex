\documentclass{beamer}
\mode<presentation>
\usepackage[english,russian]{babel}
\usepackage[utf8]{inputenc}
\usepackage[T2A]{fontenc}
\usetheme{Warsaw}

\parindent=8pt

\usepackage{ragged2e}
\justifying
\renewcommand{\raggedright}{\leftskip=0pt \rightskip=0pt plus 0cm}

\setbeamercovered{transparent}

\begin{document}

\title[Подготовка растровых картографических данных]{Подготовка растровых картографических данных к использованию в стандартизированных навигационных и геоинформационных системах}
\author{Пётр Смирнов, Даниил Клюев}
\institute{ГБОУ Физико-математический лицей №239\\
Центрального района Санкт-Петербурга\\
Научный руководитель: Чихачёв Кирилл Борисович}
\date{Апрель 2013}

\frame{\titlepage}

\begin{frame}{Описание проблемы}
Территория России представлена советскими топографическими картами в проекции Гаусса-Крюгера "--- нельзя <<склеить>> по углу.
\end{frame}

\begin{frame}{Проекция Гаусса-Крюгера}
Описание проекции
\end{frame}

\begin{frame}{Проекция Меркатора}
Описание проекции
\end{frame}

\begin{frame}{map-файл}
Описание map-файла
\end{frame}

\begin{frame}{Постановка задачи}
\begin{enumerate}
\item Перепроецирование карты из проекции Гаусса-Крюгера в проекцию Меркатора по имеющейся информации о нескольких точках (уточнить?)
\pause
\item Масштабирование и разбивка на тайлы (которые затем можно использовать)
\pause
\item Создание новых map-файлов
\end{enumerate}
\end{frame}

\begin{frame}{Сферический список литературы в вакууме}
 \begin{thebibliography}{10}
\beamertemplatebookbibitems
\bibitem{LurkHorse}
{\sc Lurkmore}, {\em Сферический конь в вакууме}.
\bibitem{AbsHorse}
{\sc Абсурдопедия}, {\em Сферический конь в вакууме}.
\end{thebibliography}
\end{frame}

\end{document}